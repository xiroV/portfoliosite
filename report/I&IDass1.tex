\documentclass[a4paper,titlepage,fleqn,12pt]{article}

\usepackage[utf8]{inputenc}
\usepackage[T1]{fontenc}
\usepackage[english]{babel}
\usepackage{color}
\usepackage{float}
\usepackage{fancyvrb}
\usepackage{amssymb}
\usepackage{amsmath}
\usepackage{listings}

\usepackage{comment}

\usepackage{graphicx}
\usepackage{ulem}
\usepackage{pdfpages}

\DeclareGraphicsExtensions{.png}

\definecolor{dkgreen}{rgb}{0,0.45,0}
\definecolor{gray}{rgb}{0.5,0.5,0.5}
\definecolor{mauve}{rgb}{0.30,0,0.30}

\lstset{frame=tb,
  language=Java,
  aboveskip=3mm,
  belowskip=3mm,
  showstringspaces=false,
  columns=flexible,
  basicstyle={\small\ttfamily},
  numbers=left,
  numberstyle=\footnotesize,
  keywordstyle=\color{dkgreen}\bfseries,
  commentstyle=\color{dkgreen},
  stringstyle=\color{mauve},
  frame=single,
  breaklines=true,
  breakatwhitespace=false
  tabsize=1
}


\begin{document}

\begin{titlepage}
	\begin{center}
	\includegraphics[scale=1.5,page=7]{sdu_logos.pdf}~\\[0.5cm]
	\textsc{\Large{SydDansk Universitet - Mærsk Merkini Møller Institutet}} \\[0.2cm]
	\rule{12cm}{1pt} \\[0.4cm]
	{ \huge \bfseries Interaktion og interaktions design, efterår 14, Projekt del 1 \\[0.4cm] }
	\rule{12cm}{1pt} \\[1.5cm]
	
	\begin{minipage}{0.4\textwidth}
		\begin{flushleft} \large
			\textit{Author:}\\
			Morten Rovelt Hansen\\
			Brian Pedersen\\
			Steven Gøhler\\
		\end{flushleft}
	\end{minipage}
	\begin{minipage}{0.4\textwidth}
		\begin{flushright} \large
			\textit{Supervisor:} \\
			Jess Uhre Rahbek
		\end{flushright}
	\end{minipage}
	
	\vfill
	
	{\large Oktober 10, 2014}
	\end{center}
	\newpage
\end{titlepage}

\section{Table Of Contents}
\tableofcontents
\newpage

\section{Problemfelt}

\textbf{Hvad vi vil lave}\\
Vi vil udarbejde en hjemmeside, der gør det nemt og brugervenligt for en jobsøgende / kommende jobsøgende, at oprette og opdaterer en online-portfolie, der er nem, hurtig og let tilgængelig at præsenterer for en arbejdgiver eller bruger. \\ 
\\
\textbf{Antagelser}\\
En antagelse kunne være at hjemmesiden ikke appelerer til alle erhvervsmæssige retninger, men mest af alt kan være brugbar for folk i kreative erhverv medgør. Det ville mest være henvist til folk i mode/design/foto verdenen, da det nemme og overskuelige aspekt mest er velegnet til at bruge sammen med fremvisninger af billeder mm.\\
\\
\textbf{vil projektet have det ønskede resultat}\\
Hvis realiseringen af projektet bliver ført ud med den ikke alt for tekniske og let designbare aspekt som vi rigtig gerne vil, kan enhver uden nogen form for kendskab til kodning, lave en let, flot og præsentabel portfolio, og projektet ville være en success.\\
\\
\section{Hvem er brugerne}
Vi prøver så vidt muligt at ramme en kreativ erhvervs gruppe der med henblik på design, selv skal kunne lave deres portfolie med klik fra musen uden spild tid eller kendskab til kodning. Den visuelle opsætning af siden skal appelerer til folk der hurtigt vil have en oversigt over en persons projekter, som nemt er til at skabe et overblik over.\\
\\
\section{Hvad er brugernes behov}
En nem og hurtig portfolie til at vise frem, uden at skulle lave en hel hjemmeside selv. En side hvor man kan fremvise sin kunnen, sine personlige oplysninger og ting man gerne vil have en anden person skal se, som personen hurtigt og visuelt nemt kan danne sig et overblik over.\\
\\
\section{Hvilke krav stiller dette til vores løsning}
-Vores side skal være nemt tilgængelig.\\
-Vores layout skal visuelt være overskueligt.\\
-Vores undersider skal være navnelinket.\\
-Vores side skal være nemt navigerlig.\\
-Vores design skal være statisk.\\
-Vores side skal være fri for andet end relevante oplysninger, men stadig være visuel flot og funktionel.\\
\\
\section{konceptuel model}
\section{Personas}
\textbf{person 1}\\
Navn,Alder: Lisa, 24\\
Beskæftigelse:	Design studerende på universitetet\\
Teknisk følelse/attitude: glad for teknologi, og fremadrettet design mæssigt.\\
Baggrrund: lisa er en frisk ung pige der til daglig studerer på universitet, og har en bred omgangskreds. lisa drømmer om at kunne afslutte universitet og komme ud og prøve kræfter med erhvervslivet men har ikke rigtig lavet andet erhvervsmæsigt end at sidde ved kassen i fakta. hun bor alene i byen uden andre. 
\\\\
\textbf{person 2}\\
Navn,Alder: Henning, 22\\
Beskæftigelse: Webintegrater freelance arbejder \\
Teknisk følelse/attitude: høj teknisk viden.
\\\\
\textbf{person 3}\\
Navn,Alder: George, 38\\
Beskæftigelse: Travl selvstændig erhvervsdrivende\\
Teknisk følelse/attitude: ikke særlig teknisk anlagt og gider ikke computere.
\\\\
\textbf{person 4}\\
Navn,Alder: Betina, 45\\
Beskæftigelse: Ansættelses medarbejder i større firma\\
Teknisk følelse/attitude: bruger behov, bruger computer på arbejde.
\\\\

\end{document}